\documentclass{article}
\usepackage{graphicx} % Required for inserting images
\usepackage{authblk} % Required for author affiliations
\usepackage{indentfirst} % Indent first paragraph of sections
\usepackage{amssymb} % For mathematical symbols
\usepackage{amsthm} % For theorem environments
\usepackage{amsmath} % For advanced math typesetting
\usepackage[hidelinks]{hyperref}
\usepackage{enumitem}
\usepackage{pgfplots} % For plots
\pgfplotsset{compat=1.18} % Set compatibility level
\usepackage{tikz} % For drawing shapes
\newtheorem{theorem}{Theorem}
\newtheorem{corollary}{Corollary}[theorem]
\newtheorem{lemma}[theorem]{Lemma}
\newtheorem{definition}{Definition}
\newtheorem{problem}{Problem}
\newtheorem{solution}{Solution}
\newtheorem{example}{Example}
\newtheorem{remark}{Remark}
\begin{document}
%------- Title page   -----------
\title{PHYS 241: Signal Processing}
\author{William Homier}
\affil[1]{McGill University Physics, 3600 Rue University, Montréal, QC H3A 2T8, Canada}
\date{January \(6^{th}\), 2026}
\setcounter{Maxaffil}{0}
\renewcommand\Affilfont{\itshape\small}
\maketitle

%------- Abstract -----------
\noindent\rule{\textwidth}{0.4pt}
\thispagestyle{empty}
\begin{abstract}
\end{abstract}
\noindent\rule{\textwidth}{0.4pt}
\clearpage

%------- Table of Contents -----------
\thispagestyle{empty}
\hypersetup{
    citecolor=black,
    filecolor=black,
    linkcolor=black,
    urlcolor=black
}
\tableofcontents
\clearpage

%------- introduction -----------
\setcounter{page}{1}
\section{Introduction}
\section{Prerequisite knowledge}
\section{Basics \& Voltage, Current and Resistance}
\subsection{Signal Types}
\subsubsection{Digital Signal}
\begin{definition}
    A discretely sampled signal with a sequence of quantized values.
\end{definition}
\subsubsection{Analogue}
\begin{definition}
    A continuous signal (e.g., in time) representing (analogous to) some other quantity.
\end{definition}
\begin{example}
    Examples of analogue devices and computers are:
    \begin{itemize}
        \item thermometers
        \item sextants
        \item tide-predicting machine
    \end{itemize}
\end{example}
\subsection{Circuits}
\subsubsection{DC}
\begin{definition}
    Direct Current (DC) is a form of current where voltage and current are constant over time.
\end{definition}
\paragraph{DC Offset}
We often talk about adding a \textbf{DC offset} to an AC signal.  
This means adding a constant DC value to an AC signal.  
Doing this shifts the entire signal up or down relative to the \(0\,\text{V}\) level, without changing the shape of the AC signal.

\begin{example}
    Example of a source of DC current is a battery.
\end{example}
\subsubsection{AC}
\begin{definition}
    Alternating Current (AC) is a form of current that changes over time, often in a sinusoidal manner.
\end{definition}
\begin{example}
    Example of a source of AC current is a transformer. Other examples of AC current are wall outlets.
\end{example}

\subsection{Linear Systems}
 
\subsection{Current flow}

\subsection{Ohm's Law}

\section{Circuit Theory Beyond Electronic}
\section{Capacitors \& Inductors}
\section{RC and LR Circuits with AC Driving}
\section{Impedance}
\section{RLC Circuits}
\subsection{Transient Response}
\subsection{Driven RLC Circuits}
\subsection{Power Input to RLC and Circuit Network}
\section{Circuit Networks}
\section{Fourier Series}
\section{Fourier Transforms}


\section{Appendix}

\section{Useful Links}

\end{document}