\documentclass[11pt]{article}

% ---------- Packages ----------
\usepackage[margin=1in]{geometry}
\usepackage{parskip}   % space between paragraphs, no indent
\usepackage{titlesec}
\usepackage{enumitem}
\usepackage{charter}

% ---------- Formatting ----------
\setlength{\parindent}{0pt}
\pagenumbering{gobble} % remove page numbers

% ---------- Document ----------
\begin{document}

% ==============================
% Page 1: Interests and Goals
% ==============================

% Replace your title section with this
\begin{flushleft}
    {\huge \textbf{William Homier}} \\
    \textit{Honours Physics, McGill University} \\
    \rule{\textwidth}{0.5pt}
    \small{ID: 261217258 \textbullet \ william.homier@mail.mcgill.ca \textbullet \ github.com/williamr-yc7}
\end{flushleft}

\section*{Statement of Interests and Goals}

\vspace{12pt}

I want to work on physics that turns fundamental ideas into technologies that change the world. My fascination with physics began long before McGill and grew out of an unexpected place: my love of history. Studying World War II and the Cold War, I became captivated not just by the events themselves, but by the technologies they produced. Radar, nuclear reactors, and jet engines during World War II showed me how physics could solve big, real-world problems. Then during the Cold War, the race for satellites, missiles, and space exploration showed me how physics could push humanity forward on a massive scale. I kept asking myself questions like: how could tiny particles release so much energy, will we ever colonize space, and how did breakthroughs in electronics and communications change the world? Those questions drew me to physics to understand how science drives innovation. My dream is to contribute to science in a way as impactful as the breakthroughs from those decades, helping push knowledge and technology forward for everyone.

At McGill, as an Honours Physics student, I have immersed myself in both the theory and hands-on sides of the field. My courses have helped me develop a solid foundation in mechanics, electromagnetism, and special relativity, along with computational methods for physics. Beyond the classroom, I joined engineering design teams including the McGill Drones and Vertical Flight Society and McGill Formula Electric, where I designed, tested, and troubleshot complex systems. These experiences taught me how to combine creativity with careful planning, work effectively with a team under pressure, and keep iterating until a design performs reliably.

I am especially drawn to research that combines experiments, simulations, and fundamental physics. My interests span condensed matter, nuclear physics, plasma, and particle physics, with a growing fascination for cosmology and dark matter. Projects that explore these areas excite me because they challenge me to think critically, solve difficult problems, and produce results that have real impact. I enjoy working on questions where I can see the direct effect of my work in the lab, on a computer, or in the field.

Through a summer research position, I want to contribute immediately by applying my experience in electronics, programming, and experimental design to active projects. I am eager to gain further experience in field testing, instrumentation development, and data analysis while sharpening my problem-solving and collaboration skills. In the long term, I plan to continue into graduate studies and a career in physics research, where I can combine theoretical insight with experimental innovation and help push the boundaries of our understanding of the universe.

\newpage

% ==============================
% Page 2: Project Preferences
% ==============================

% Replace your title section with this
\begin{flushleft}
    {\huge \textbf{William Homier}} \\
    \textit{Honours Physics, McGill University} \\
    \rule{\textwidth}{0.5pt}
    \small{ID: 261217258 \textbullet \ william.homier@mail.mcgill.ca \textbullet \ github.com/williamr-yc7}
\end{flushleft}

\section*{Preferred Summer Research Projects}

\vspace{12pt}

\textbf{1. Autonomous antenna station development for ALBATROS}

The ALBATROS project's focus on the "dark ages" of the universe is a fascinating marriage of cosmology and rugged experimentalism. Having worked with McGill Drones and Formula Electric, I am deeply familiar with the challenges of making electronics survive and perform in demanding environments. I am particularly interested in the challenge of autonomous operation in the Arctic where reliability is paramount. I can contribute immediately to the development of calibration electronics and the upgraded readout system by leveraging my experience in system troubleshooting and hardware integration. Furthermore, my background in C\# and Python positions me to contribute to the "quicklook" analysis tools and software upgrades necessary for field testing at Uapishka Station. I am eager to apply my hands-on design team mentality to help ALBATROS map the low-frequency sky.

\vspace{12pt}

\textbf{2. 3D Hydrodynamic Modeling of Relativistic Heavy Ion Collisions of Deformed Nuclei}

This project bridges my interest in the high-energy origins of the universe with my passion for high-performance computing. The opportunity to work with the MUSIC hydrodynamics code is a perfect fit for my programming background. I have experience across a wide stack of languages including C++, Python, and SQL, which allows me to not only run simulations but understand and optimize the underlying numerical workflow. I am specifically interested in how the "shape space" of colliding nuclei (from footballs to pear-shapes) dictates the final momentum-space anisotropy. My ability to write robust, efficient code will allow me to systematically vary shape parameters and analyze the resulting physical data with high precision, assisting Dr. Wu and Prof. Jeon in probing the properties of Quark-Gluon Plasma.

\vspace{12pt}

\textbf{3. THz-driven point projection electron microscopy}

The Quantum Dynamics Laboratory represents the exact kind of frontier technology that drew me to physics via my interest in historical breakthroughs. Designing a collimator to switch between real-space imaging and diffraction mode is a task that directly utilizes my experience with NX CAD software from McGill Formula Electric. I am excited by the prospect of using COMSOL field simulations to verify local field strengths—a task that requires the same iterative design-and-test mindset I applied to drone systems. My technical versatility, ranging from mechanical assembly to C-based programming, will allow me to assist in building the radial micro-electron lens and testing electron deflection within the vacuum chamber. I am highly motivated to contribute to the commissioning of this state-of-the-art instrument and to participate in the TrUST NSERC CREATE program.


\end{document}
