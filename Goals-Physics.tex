\documentclass[11pt]{article}

% ---------- Packages ----------
\usepackage[margin=1in]{geometry}
\usepackage{parskip}   % space between paragraphs, no indent
\usepackage{titlesec}

% ---------- Formatting ----------
\setlength{\parindent}{0pt}
\pagenumbering{gobble} % remove page numbers

% ---------- Document ----------
\begin{document}

% ==============================
% Page 1: Interests and Goals
% ==============================

\begin{center}
    {\Large \textbf{Statement of Interests and Goals in Physics}}\\
    \vspace{4pt}
    William Homier
\end{center}

\vspace{12pt}

I am an undergraduate student in Honours Physics at McGill University with a strong academic background and a deep interest in understanding how fundamental physical principles govern real-world systems. Through my coursework, I have built a solid foundation in mechanics, electromagnetism, quantum physics, and the mathematical methods used throughout physics.

My primary interests lie in experimental and computational studies of matter at small and extreme scales, particularly in condensed matter, nuclear, plasma, and particle physics, with additional interest in cosmology and dark matter. I am drawn to projects that combine theory, simulation, and lab work, allowing models to be tested and refined against data.

Outside the classroom, I have participated in student engineering design teams including the McGill Drones and Vertical Flight Society and McGill Formula Electric. Through these teams, I gained exposure to collaborative design processes, basic testing, and technical problem-solving in applied engineering settings. Working in these environments helped me become more comfortable with hands-on systems and teamwork.

Through a summer research position, I hope to further develop my research skills, gain experience with advanced experimental and computational techniques, and contribute meaningfully to an active research group. In the long term, I aim to pursue graduate studies in physics and build a career developing new scientific knowledge and technologies through research.

\newpage

% ==============================
% Page 2: Project Preferences
% ==============================

\begin{center}
    {\Large \textbf{Preferred Summer Research Projects}}\\
    \vspace{4pt}
    William Homier
\end{center}

\vspace{12pt}

\textbf{1. Autonomous antenna station development for ALBATROS}

I chose this project because I'm fascinated by the possibility of exploring the universe's dark ages and mapping the low-frequency sky. I'm especially curious about how the autonomous antenna stations work and how they can detect signals from such early times. My experience with electronics and lab work will let me contribute to building and testing hardware and software for the stations. I hope to gain more experience in experimental setup, field testing, and collaborative research while seeing how the data comes together. I'm also excited by the potential connection to dark matter studies, which makes this project even more interesting to me.

\vspace{12pt}

\textbf{2. 3D Hydrodynamic Modeling of Relativistic Heavy Ion Collisions of Deformed Nuclei}

This project interests me because it blends computational modeling with fundamental nuclear physics, giving insight into the behavior of matter under extreme conditions. My programming experience and mathematical background will allow me to engage with the hydrodynamic simulations and analyze the resulting data effectively. I am motivated to explore how the shape of colliding nuclei affects the evolution of Quark-Gluon Plasma and to develop skills in numerical simulations and data interpretation. Participating in this project would provide valuable exposure to theoretical and computational research methods.

\vspace{12pt}

\textbf{3. THz-driven point projection electron microscopy}

I am drawn to this project because it offers a unique opportunity to work on a cutting-edge ultrafast electron microscopy instrument, combining precision engineering with quantum dynamics. My background in experimental physics and lab-based problem solving will help me contribute to the design, simulation, and testing of the electron collimator. I hope to gain experience in ultrafast imaging techniques, COMSOL simulations, and hands-on instrumentation development. This project would provide a rare chance to engage with state-of-the-art research and develop skills relevant to both experimental and applied physics.


\end{document}
