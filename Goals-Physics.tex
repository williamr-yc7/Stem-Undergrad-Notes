\documentclass[11pt]{article}

% ---------- Packages ----------
\usepackage[margin=1in]{geometry}
\usepackage{parskip}   % space between paragraphs, no indent
\usepackage{titlesec}
\usepackage{enumitem}
\usepackage{charter}

% ---------- Formatting ----------
\setlength{\parindent}{0pt}
\pagenumbering{gobble} % remove page numbers

% ---------- Document ----------
\begin{document}

% ==============================
% Page 1: Interests and Goals
% ==============================

% Replace your title section with this
\begin{flushleft}
    {\huge \textbf{William Homier}} \\
    \textit{Honours Physics, McGill University} \\
    \rule{\textwidth}{0.5pt}
    \small{ID: 261217258 \textbullet \ william.homier@mail.mcgill.ca \textbullet \ github.com/williamr-yc7}
\end{flushleft}

\section*{Statement of Interests and Goals}

\vspace{12pt}

I want to work on physics that turns fundamental ideas into technologies that change the world. My interest in physics grew through studying history, where events like World War II and the Cold War showed how scientific breakthroughs such as nuclear energy, jet engines, rockets, satellites, and early computing pushed humanity forward on a massive scale. I became fascinated by how abstract concepts, like the atom or the structure of the universe, could lead to discoveries as profound as nuclear power, space exploration, and the Big Bang. That curiosity made me wonder what overlooked mysteries today, such as dark matter, might unlock tomorrow. These questions drew me to physics to understand how deep theory becomes real innovation, and my goal is to contribute research that advances both knowledge and technology for society.

At McGill, as an Honours Physics student, I built a strong foundation in mechanics, electromagnetism, special relativity, and computational methods. I also participated in engineering design teams including the McGill Drones and Vertical Flight Society and McGill Formula Electric, where I gained exposure to system design, testing, and collaborative hardware work. These experiences strengthened my practical problem solving skills and my ability to work effectively in teams.

I am especially drawn to research that blends experiment, simulation, and fundamental theory, with primary interests in nuclear and particle physics, a strong curiosity about plasma physics, and a growing fascination with cosmology and dark matter. Projects that explore these areas interest me because they challenge me to think critically, solve difficult problems, and produce results that have real impact.

Through a summer research position, I hope to contribute using my background in electronics, programming, and experimental design while gaining deeper experience in instrumentation and data analysis. I plan to continue into graduate studies and pursue a career in physics research.

\newpage

% ==============================
% Page 2: Project Preferences
% ==============================

% Replace your title section with this
\begin{flushleft}
    {\huge \textbf{William Homier}} \\
    \textit{Honours Physics, McGill University} \\
    \rule{\textwidth}{0.5pt}
    \small{ID: 261217258 \textbullet \ william.homier@mail.mcgill.ca \textbullet \ github.com/williamr-yc7}
\end{flushleft}

\section*{Preferred Summer Research Projects}

\vspace{12pt}

\textbf{1. Autonomous antenna station development for ALBATROS}

The ALBATROS project's focus on mapping the low-frequency radio sky aligns with my interest in experimental astrophysics. I have read parts of the ALBATROS instrument paper and related reports, and I am fascinated by how the array combines autonomous stations and interferometry to collect data in a remote, radio-quiet environment. While the project focuses on developing and testing hardware and software for the stations, I am motivated by broader questions about the dark ages of the universe and how observations of the 21 cm signal could eventually reveal early matter distribution and the role of dark matter. Through my coursework and participation in engineering design teams, I have worked on team projects and done hands-on system testing, which sparked my interest in experimental work. I hope to contribute to testing and software while gaining hands-on experience.

\vspace{12pt}

\textbf{2. 3D Hydrodynamic Modeling of Relativistic Heavy Ion Collisions of Deformed Nuclei}

This project caught my eye because it connects high-energy physics to the conditions of the early universe. I am curious about the "shape space" of these collisions and how a nucleus shaped like a pear versus a football totally changes the particle momentum in the Quark-Gluon Plasma. I'd love to use my experience in C++, Python, and SQL to run these simulations and see how those shape parameters actually play out in real physical systems. I am looking forward to getting involved in the numerical modeling and seeing how the theoretical side of things works in practice.

\vspace{12pt}

\textbf{3. THz-driven point projection electron microscopy}

I'm interested in experimental physics, especially building and testing instruments that measure very fast processes. I would like to help design and test the collimator, use simulations to understand how the electrons are guided, and then work in the lab to see how the setup performs in practice. I enjoy both coding for data analysis and hands-on work like assembling and troubleshooting equipment, and I'm looking forward to learning how an ultrafast electron microscope is built and used.


\end{document}
